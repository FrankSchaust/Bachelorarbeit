\section{Motivation}

Das Thema der Bachelorarbeit ist die Adaption von Machine Learning (zu Deutsch: Maschinelles Lernen) Algorithmen zur Vorhersage von Gefechts-Ausgängen zweier feindlicher Armeen in dem Echtzeit-Strategiespiel StarCraft II. Das Ziel der Arbeit ist die Untersuchung, ob existierende Image Classification (zu Deutsch: Bildklassifikation) Algorithmen für komplexe Aufgaben in einer atypischen Domäne genutzt werden können. Kernbestandteil ist daher der Vergleich bestehender Architekturen auf dieser neuen Domäne. Die Domäne unterscheidet sich von konventionellen Bildformaten, da die Eingabedaten in Form von domänenspezifischen Eigenschaftsmatrizen vorliegen. Es werden vorwiegend Convolutional Neural Networks (folgend CNN, zu Deutsch etwa: Faltende Neuronale Netze) genutzt um Matrix-Repräsentationen von Spielzuständen zu verarbeiten und auf Grundlage derer den Ausgang der Kampfszenarien vorherzusagen. 

CNNs sind eine Machine Learning Methode, die unter anderem von den Gewinnern der ImageNet Large Scale Visual Recognition Challenge \textit{GoogLeNet} \parencite{DBLP:journals/corr/SzegedyLJSRAEVR14} genutzt wurde. In diesem Aufgabenfeld konnten CNNs im Vergleich zu anderen Image Classification Verfahren gute Resultate erzielen. Außerdem fanden CNNs in der Vergangenheit erfolgreich Anwendung in anderen Forschungsgebieten, welche keinen direkten Bezug zu Image Classification hatten, wie zum Beispiel bei der Entdeckung von medizinischen Wirkstoffen (AtomNet \parencite{DBLP:journals/corr/WallachDH15}).

Gefechte in StarCraft II bestehen aus zwei oder mehr Armeen. Jede Armee wird von einem unterschiedlichen Spieler kontrolliert, dessen Ziel es ist mit den Fähigkeiten seiner eigenen Einheiten sämtliche Einheiten des Gegners zu zerstören. Gefechte beinhalten komplexe Interaktionen, da jede Einheit ihre Position auf dem Schlachtfeld ändern und feindliche Einheiten angreifen und schlussendlich zerstören kann. Erfahrene menschliche Spieler können Gefechts-Ausgänge solcher Aufeinandertreffen im Allgemeinen mit einer angemessenen Wahrscheinlichkeit vorhersagen.

Um den Zustand des Spiels darzustellen werden Matrix-Repräsentationen genutzt. Jede Matrix repräsentiert einen anderen Aspekt (wie zum Beispiel Einheiten-Gesundheit, Einheiten-Typ, usw.) des Spiel-Zustandes in Bezug auf die Position auf dem Spielfeld. Gefechts-Vorhersage kann als Image Classification Aufgabe gesehen werden, da die genutzten Matrizen in einer ähnlichen Form klassifiziert werden können wie Bilder. Es werden Matrizen als EIngabe genutzt und es werden diskrete Werte als Ausgabe erhalten. Der Unterschied zwischen Bildern und Matrix-Repräsentation ist, dass jedes Bild explizit ein  oder mehrere Objekte zeigt, welche klassifiziert werden, während jede Matrix-Repräsentation eine implizierte Bedeutung hat. Die impliziten Bedeutungen zu lernen und zu kombinieren ist eine interessante Aufgabe für eine CNN-Architektur.

Der Versuch das Verhalten von Computerspielen vorherzusagen ist interessant, weil sie eine hervorragende Umgebung bieten um die Leistungsfähigkeit von Machine Learning Methoden zu testen. Sie stellen eine kontrollierbare Menge an Umweltfaktoren bereit -- im Fall von StarCraft~II die Anzahl der Einheiten, ihre Positionen und Kampfstärke -- welche genutzt werden kann um beliebig viele Trainingsdaten von kontrollierbarer Schwierigkeit zu erzeugen. Der Zustand des Spiels und der Ausgang des Gefechts sind klar zu bemessen und es gibt konkrete Regeln, die der Algorithmus lernen muss, um zuverlässige Vorhersagen zu treffen, wie zum Beispiel: \textit{Ein Marine greift mit X Schadenspunkten an}, oder \textit{Wenn die Lebenspunkte einer Einheit auf 0 oder tiefer fallen, stirbt sie}. Die Schwierigkeit für den Algorithmus liegt darin die richtigen Schlüsse zu ziehen, ohne explizites Wissen über die zugrundeliegenden Regeln des Spiels zu haben. Neben der Eignung als Machine Learning Problem, wird ein erfolgreicher Vorhersage-Algorithmus benötigt um künstliche Intelligenzen (KIs) für StarCraft~II zu implementieren. Damit KIs das Spiel auf einem hohen Niveau spielen können, müssen sie in der Lage sein, den Ausgang eines Gefechtes vorherzusagen um abzuwägen, ob sie den Kampf eingehen können, oder den Rückzug wählen müssen. 

In dieser Bachelorarbeit werden die betreffenden theoretischen Grundlagen für diese Problemstellung in Sektion \ref{th} zusammengetragen. Zudem liefert Sektion \ref{SC2} eine formalisierte Darstellung der Spielumgebung von StarCraft~II. Verwandte Arbeiten wurden auf mögliche Beiträge zu diesem Thema analysiert und in Sektion \ref{VerwandteArbeiten} zusammengefasst. Es wurde ein Framework geschaffen, welches basierend auf dem in StarCraft~II enthaltenen Map-Editor die Generierung beliebig vieler Trainingsdaten sowohl in Form von Spieldaten in Matrix-Repräsentation als auch als Screenshots in gängigen Bildformaten ermöglicht. Das Framework wird in Sektion \ref{datagen} genauer beschrieben. Zudem wurden bestehende Architekturen für die fallspezifischen Eingabematrizen adaptiert. Die genutzten Architekturen werden in Sektion \ref{Archs} genauer beschrieben und in Sektion \ref{Resultate} analysiert und verglichen. 

\subsection{StarCraft~II}
\label{SC2}

StarCraft~II ist ein Echtzeit-Strategiespiel entwickelt und veröffentlicht von Blizzard Entertainment. Es ist überwiegend deterministisch\footnotemark und das Ziel ist es alle feinlichen Einheiten auf einer Karte zu vernichten ohne selbst dabei vernichtet zu werden.  

Zu Beginn eines jeden Spiels kann der Spieler zwischen drei Rassen wählen. Jeder Spieler kontrolliert eine Menge an Einheiten U = \{$u_1, u_2, ... u_n$\}.

Das Spiel läuft im Allgemeinen in Echtzeit ab, kann allerdings auf Zeitschritte abstrahiert werden, wo in jedem Zeitschritt die gegebenen Befehle den kommenden Zeitschritt beeinflussen. Daher kann der Spielablauf als eine diskrete Serie von Zuständen repräsentiert werden. Intern läuft das Spiel mit einer Rate von 20 Zuständen pro Sekunde. 

\footnotetext{Die hauptsächlichen Quellen für Zufälligkeiten sind laut DeepMind \textit{Angriffsgeschwindigkeit} und \textit{Aktualierungsreihenfolge}. Diese Zufälligkeiten können durch das manuelle setzen eines Zufallswertes ausgemerzt werden  \parencite{DBLP:journals/corr/abs-1708-04782} und in die Berechnungen eingearbeitet werden.}

Jede Einheit besitzt eine allgemeine Beschreibung die ihr einen Typ $t(u_i)$ zuweist. $t(u_i)$ kann einer von drei Typen sein $t(u_i) \in$ \{ Bodeneinheit, Lufteinheit, kleine Bodeneinheit \}. Zusätzlich kann jede Einheit mit zwei weiteren Attributen $at(u_i)$, $st(u_i)$ versehen werden, die ihre weiteren Eigenschaften spezifizieren. Eine Einheit kann eine leichte oder eine gepanzerte Einheit sein ($at(u_i)$) und außerdem dem biologischen oder mechanischen Typ angehören ($st(u_i)$). Es gibt vereinzelt Ausnahmen in denen Einheiten keinem Typ oder beiden Typen angehören.

Einheiten bewegen sich auf einer 2D Karte und können in Abhängigkeit von der Zeit einer Position zugeordnet werden. Die Karte kann als Matrix mit Dimensionen $H \times B$, $H,B \in \mathbb{N}$ definiert werden, wodurch die Position einer Einheit als zweidimensionaler Punkt $p^t(u_i)$ = $(x,y)$ verstanden werden kann, bei dem $t$ die Spielzeit darstellt und $x \leq B, y \leq H$ gilt.

Jede Einheit verfügt über ein Attribut Lebenspunkte $hp^t(u_i)$, eine Bewegungsrate $mr(u_i)$ und eine Sichtweite $sr(u_i)$. Lebenspunkte sind von der Spielzeit abhängig und können sich während dem Verlauf des Spiel ändern. Wenn $\exists t$ indem $hp^t(u_i) \leq 0$ gilt, so wird die Einheit for alle folgenden Zeitschritte $t' > t$ als tot behandelt und ist nicht mehr in der Lage ins Spielgeschehen einzugreifen. Die Bewegungsrate bestimmt wie schnell sich eine Einheit über die Spielkarte bewegen kann. Die Sichtweite bestimmt den Radius um eine freundliche Einheit herum in dem die Karte aufgedeckt wird. Ein Spieler kann nur jene gegnerischen Einheiten sehen, welche sich im Sichtbereich von mindestens einer freundlichen Einheit aufhalten. Zusätzlich verfügen Einheiten über weitere Attribute, welche ihre Kampfstärke definieren: Schaden $d(u_i)$, Abklingzeit $cd(u_i)$ und Waffenreichweite $wr(u_i)$. Der Schadenswert wird genutzt um den Schaden zu berechnen, den eine Einheit mit einem einzelnen Schuss verursacht. Abklingzeit bestimmt wie lange eine Einheit warten muss bevor sie einen weiteren Schuss abgeben darf und wird in Sekunden bemessen. Die Waffenreichweite ist der Radius eines Kreises in dessen Fläche die Einheit Angriffe ausführen kann. Der Mittelpunkt des Kreises ist immer die Position der Einheit. 

Das Schadensattribut kann durch eine Menge an möglichen Zieltypen $tt(u_i)$ erweitert werden. Eine Einheiten können zum Beispiel lediglich Bodeneinheiten angreifen, während andere nur Lufteinheiten angreifen können. Daher kann die Menge der angreifbaren Einheiten $tu(u_i)$ definiert werden als Menge aller gegnerischen Einheiten $EU$ für die gilt: \{$x \in EU : t(x) \in tt(u_i)$\}. Einige Einheiten verfügen außerdem über ein Attribut Bonusschaden $bd^{at}(u_i)$, welches auf dem at-Attribut der attackierten Einheit beruht. Die Angriffe von Einheiten können zusätzlich verbessert werden, indem man bei den Einheiten die Verbesserungen ausbildet. Für jede Angriffsverbesserung erhält die Einheit einen Bonus von $d(u_i) \div 10$ - jedoch mindestens 1 - auf ihren Schaden. 

Viele Einheiten sind in der Lage Fähigkeiten $ab(u_i)$ zu nutzen. Die Fähigkeiten unterscheiden sich in ihren Effekten und Nutzungsmöglichkeiten. Einige können aktiv und manuell vom Spieler eingesetzt werden, andere sind passiv oder werden von bestimmten Konditionen im Spiel ausgelöst. Zauber sind eine besondere Form der Fähigkeiten und kosten die Einheiten Energie. Obwohl die intelligente Nutzung von Fähigkeiten und Zaubern in Mensch-gegen-Mensch Gefechten eine signifikante Rolle spielen, werden sie in dieser Arbeit vernachlässigt, da sie den simulierten Armeekommandeuren die Fähigkeit des automatisierten Lernens abverlangen würden. 