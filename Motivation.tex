\section{Motivation}

Das Thema der Bachelorarbeit ist die Adaption von Algorithmen aus dem Machine-Learning (zu Deutsch: \textit{Maschinelles Lernen}) zur Vorhersage von Gefechts-Ausgängen zweier feindlicher Armeen in dem Echtzeit-Strategiespiel StarCraft~II. Das Ziel der Arbeit ist die Untersuchung, ob existierende Algorithmen, die im Bereich der Image-Classification (zu Deutsch: \textit{Bildklassifikation}) Anwendung finden, für komplexe Aufgaben in einer atypischen Domäne genutzt werden können. Kernbestandteil ist daher der Vergleich bestehender Architekturen auf dieser neuen Domäne. Die Domäne unterscheidet sich von konventionellen Bildformaten, da die Eingabedaten in Form von domänenspezifischen Eigenschaftsmatrizen vorliegen. Es werden vorwiegend Convolutional Neural Networks (folgend CNNs, zu Deutsch etwa: \textit{Faltende Neuronale Netze}) genutzt um Matrix-Repräsentationen von Spielzuständen zu verarbeiten und auf Grundlage derer den Ausgang der Kampfszenarien vorherzusagen. 

CNNs sind eine Machine-Learning-Methode, die unter anderem von den Gewinnern der ImageNet Large Scale Visual Recognition Challenge \textit{GoogLeNet} \parencite{DBLP:journals/corr/SzegedyLJSRAEVR14} genutzt wurde. In diesem Aufgabenfeld konnten CNNs im Vergleich zu anderen Image Classification Verfahren gute Resultate erzielen. Außerdem fanden CNNs in der Vergangenheit erfolgreich Anwendung in anderen Forschungsgebieten, welche keinen direkten Bezug zu Image Classification hatten, wie zum Beispiel bei der Entdeckung von medizinischen Wirkstoffen (AtomNet \parencite{DBLP:journals/corr/WallachDH15}).

Gefechte in StarCraft II bestehen aus zwei oder mehr Armeen. Jede Armee wird von jeweils einem Spieler kontrolliert, dessen Ziel es ist mit den Fähigkeiten seiner eigenen Einheiten sämtliche Einheiten des Gegners zu zerstören. Gefechte beinhalten komplexe Interaktionen, da jede Einheit ihre Position auf dem Schlachtfeld ändern und feindliche Einheiten angreifen und schlussendlich zerstören kann. Erfahrene menschliche Spieler können Gefechts-Ausgänge solcher Aufeinandertreffen im Allgemeinen mit einer guten Wahrscheinlichkeit vorhersagen.

Um den Zustand des Spiels darzustellen werden Matrix-Repräsentationen genutzt. Jede Matrix repräsentiert einen anderen Aspekt (wie zum Beispiel Einheiten-Gesundheit, Einheiten-Typ, usw.) des Spiel-Zustandes in Bezug auf die Position auf dem Spielfeld. Die Aufgabe der Gefechts-Vorhersage kann als Image-Classification-Aufgabe gesehen werden, da die genutzten Matrizen in einer ähnlichen Form klassifiziert werden können wie Bilder. Es werden Matrizen als Eingabe genutzt und es werden diskrete Werte als Ausgabe erhalten. Der Unterschied zwischen Bildern und Matrix-Repräsentation ist, dass jedes Bild explizit ein  oder mehrere Objekte zeigt, welche klassifiziert werden, während jede Matrix-Repräsentation eine implizierte Bedeutung hat. Die impliziten Bedeutungen zu lernen und zu kombinieren ist eine interessante Aufgabe für eine CNN-Architektur.

Der Versuch das Verhalten von Computerspielen vorherzusagen ist interessant, weil sie eine hervorragende Umgebung bieten um die Leistungsfähigkeit von Machine-Learning-Methoden zu testen. Sie stellen eine kontrollierbare Menge an Umweltfaktoren bereit -- im Fall von StarCraft~II die Anzahl der Einheiten, ihre Positionen und Kampfstärke -- welche genutzt werden kann um beliebig viele Trainingsdaten von kontrollierbarer Schwierigkeit zu erzeugen. Der Zustand des Spiels und der Ausgang des Gefechts sind klar zu bemessen und es gibt konkrete Regeln, die ein Algorithmus lernen muss, um zuverlässige Vorhersagen zu treffen, wie zum Beispiel: \textit{Ein Marine greift mit X Schadenspunkten an}, oder \textit{Wenn die Lebenspunkte einer Einheit auf 0 oder tiefer fallen, stirbt sie}. Die Schwierigkeit für einen Algorithmus liegt darin die richtigen Schlüsse zu ziehen, ohne explizites Wissen über die zugrundeliegenden Regeln des Spiels zu haben. Neben der Eignung als Machine-Learning-Problem, wird ein erfolgreicher Vorhersage-Algorithmus benötigt um künstliche Intelligenzen (KIs) für StarCraft~II zu implementieren. Damit KIs das Spiel auf einem hohen Niveau spielen können, müssen sie in der Lage sein, den Ausgang eines Gefechtes vorherzusagen um abzuwägen, ob sie den Kampf eingehen können, oder den Rückzug wählen sollten. 

\subsection{StarCraft~II}
\label{SC2}

StarCraft~II ist ein Echtzeit-Strategiespiel entwickelt und veröffentlicht von Blizzard Entertainment. Es ist überwiegend deterministisch\footnotemark. Das Ziel ist es alle feindlichen Einheiten auf einer Karte zu vernichten ohne selbst dabei vernichtet zu werden. Im Rahmen der Arbeit wird lediglich auf die Gefechte des Spiels Bezug genommen. Von anderen Aspekten des Spiels, wie Basenbau oder Ressourcen-Management wird abstrahiert, um das Szenario einfach zu halten.

Das Spiel wird mit 20 Zuständen pro Sekunde berechnet, läuft jedoch für den menschlichen Spieler wie in Echtzeit ab. Durch die Berechnung der 20 Zustände pro Sekunde, kann auf Zeitschritte abstrahiert werden, wobei in jedem Zeitschritt die gegebenen Befehle den kommenden Zeitschritt beeinflussen. Daher kann der Spielablauf als eine diskrete Serie von Zuständen repräsentiert werden.

\footnotetext{Die hauptsächlichen Quellen für Zufälligkeiten sind laut DeepMind \textit{Angriffsgeschwindigkeit} und \textit{Aktualierungsreihenfolge}, welche durch einen Random Seed (zu Deutsch etwa: \textit{Startwert}) initialisiert werden. Diese Zufälligkeiten können durch das manuelle setzen eines festen Random Seeds umgangen werden \parencite{DBLP:journals/corr/abs-1708-04782}.}

Zu Beginn eines jeden Spiels kann der Spieler zwischen drei Rassen wählen. Jeder Spieler kontrolliert eine Menge an Einheiten U = \{$u_1, u_2, ... u_n$\}. 

Jede Einheit besitzt eine allgemeine Beschreibung die ihr einen Typ $t(u_i)$ zuweist. $t(u_i)$ kann einer von zwei Typen sein $t(u_i) \in$ \{ Boden, Luft \}. Zusätzlich kann jede Einheit mit einer Menge an Attributen $a(u_i)$ versehen werden, die ihre weiteren Eigenschaften spezifizieren und aus der Menge \{ Leicht, Gepanzert, Biologisch, Mechanisch \} zu wählen sind.

Einheiten bewegen sich auf einer 2D-Karte und befinden sich zu jedem Zeitpunkt an einer spezifischen Position. Die Karte kann als Matrix mit Dimensionen $H \times B$, $H,B \in \mathbb{N}$ definiert werden, wodurch die Position einer Einheit als zweidimensionaler Punkt $p^t(u_i)$ = $(x,y)$ verstanden werden kann, bei dem $t$ die Spielzeit darstellt und $x \leq B, y \leq H$ mit $x \in [0,B], y \in [0,H]$ gilt.

Jede Einheit verfügt über ein Attribut Lebenspunkte $hp^t(u_i)$, eine Bewegungsrate $mr(u_i)$ und eine Sichtweite $sr(u_i)$. Lebenspunkte sind von der Spielzeit abhängig und können sich während dem Verlauf des Spiel ändern. Wenn $\exists t$ indem $hp^t(u_i) \leq 0$ gilt, so wird die Einheit for alle folgenden Zeitschritte $t' > t$ als tot behandelt und ist nicht mehr in der Lage ins Spielgeschehen einzugreifen. Die Bewegungsrate bestimmt wie schnell sich eine Einheit über die Spielkarte bewegen kann. Die Sichtweite bestimmt den Radius um eine freundliche Einheit herum in dem die Karte aufgedeckt wird. Ein Spieler kann nur jene gegnerischen Einheiten sehen, welche sich im Sichtbereich von mindestens einer freundlichen Einheit aufhalten. Zusätzlich verfügen Einheiten über weitere Attribute, welche ihre Kampfstärke definieren: Schaden $d(u_i)$, Abklingzeit $cd(u_i)$ und Waffenreichweite $wr(u_i)$. Der Schadenswert wird genutzt um den Schaden zu berechnen, den eine Einheit mit einem einzelnen Schuss verursacht. Abklingzeit bestimmt wie lange eine Einheit warten muss bevor sie einen weiteren Schuss abgeben darf und wird in Sekunden bemessen. Die Waffenreichweite ist der Radius eines Kreises in dessen Fläche die Einheit Angriffe ausführen kann. Der Mittelpunkt des Kreises ist immer die Position der Einheit. 

Das Schadensattribut kann durch eine Menge an möglichen Zieltypen $tt(u_i)$ erweitert werden. Eine Einheiten kann zum Beispiel lediglich Bodeneinheiten angreifen, während eine andere nur Lufteinheiten angreifen kann. Daher kann die Menge der angreifbaren Einheiten $tu(u_i)$ definiert werden als Menge aller gegnerischen Einheiten $EU$ für die gilt: \{$x \in EU : t(x) \in tt(u_i)$\}. Einige Einheiten verfügen außerdem über ein Attribut Bonusschaden $bd^{a}(u_i)$, welches auf den Attributen $a(u_i)$ der attackierten Einheit beruht. Die Angriffe von Einheiten können zusätzlich verbessert werden, indem man bei den Einheiten die Verbesserungen ausbildet. Für jede Angriffsverbesserung erhält die Einheit einen Bonus von $d(u_i) \div 10$ -- jedoch mindestens 1 -- auf ihren Schaden. 

Viele Einheiten sind in der Lage Fähigkeiten $ab(u_i)$ zu nutzen. Die Fähigkeiten unterscheiden sich in ihren Effekten und Nutzungsmöglichkeiten. Einige können aktiv und manuell vom Spieler eingesetzt werden, andere sind passiv oder werden von bestimmten Konditionen im Spiel ausgelöst. Zauber sind eine besondere Form der Fähigkeiten und kosten die Einheiten Energie. Obwohl die intelligente Nutzung von Fähigkeiten und Zaubern in Mensch-gegen-Mensch Gefechten eine signifikante Rolle spielen, werden sie in dieser Arbeit vernachlässigt, da sie den simulierten Armeekommandeuren die Fähigkeit des automatisierten Lernens abverlangen würden.

Eine Einheiten-Beschreibung anhand der StarCraft~II Einheit Immortal würde demnach wie folgt aussehen:
Ein \textit{Immortal} (folgend kurz I) bekommt die Attribute $a(I) = \{ Gepanzert, Mechanisch \}$ zugewiesen. Die Einheit startet bei Zeitschritt 0 mit $hp^0(I) = 200$ Lebenspunkten und erhält eine Bewegungsrate von $mr(I) = 3,15$, sowie eine Sichtweite von $sr(I) = 9$. Der Schaden liegt bei $d(I) = 20$ und die Abklingzeit zwischen jedem Angriff beträgt $cd(I) = 1,04$ Sekunden. Mit seiner Waffe kann der Immortal $wr(I) = 6$ weit angreifen. Die Menge der möglichen Zieltypen ist beim Immortal $tt(I) = \{ Boden \}$. Immortal erhalten einen Bonus gegen gepanzerte Einheiten, dieser ist spezifiziert als $bd^{Gepanzert}(I) = 30$. 

\subsection{Kontributionen}

In dieser Bachelorarbeit wird eine formalisierte Darstellung der Spielumgebung von StarCraft~II in Sektion \ref{SC2} geliefert. Weiterhin werden die wichtigsten theoretischen Grundlagen für den Bereich Machine Learning mittels Convolutional Neurak Networks in Sektion \ref{th} zusammengetragen. Im Laufe der Arbeiten wurden Verwandte Arbeite auf mögliche Beiträge zu diesem Thema analysiert und in Sektion \ref{VerwandteArbeiten} mit Blick auf ihre Herangehensweise zusammengefasst. Es wurde ein Framework geschaffen, welches basierend auf dem in StarCraft~II enthaltenen Map-Editor die Generierung beliebig vieler Trainingsdaten sowohl in Form von Spieldaten in Matrix-Repräsentation ermöglicht. Das Framework wird in Sektion \ref{datagen} genauer beschrieben und kann durch das Erstellen eigener Karten beliebig erweitert werden. Zudem wurden bestehende Architekturen für die fallspezifischen Eingabematrizen adaptiert und an die gegebenen Rechenmöglichkeiten angepasst. Die genutzten Architekturen werden in Sektion \ref{Meth} genauer beschrieben. Sektion \ref{Meth} liefert zudem eine eigens implementierte Baseline, die zum Vergleich der Leistungsfähigkeit der Architekturen dient. Außerdem gibt die Sektion \ref{Eval} einen Überblick über die angewandten Evaluationsmetriken und die Trainingsmethodik. Sektion \ref{Resultate} analysiert und vergleicht die Resultate aller Netze und interpretiert diese in Bezug auf die Forschungsziele. 